% --------------------------------------------------------------------------------------------------
\section{Survival analysis methods}

% ..................................................................................................
\subsection{Identification of significant covariates}
Regarding the correlation matrix presented in figure \ref{fig_xcorr}, the \texttt{isWarm, temperature, 
windSpeed} and \texttt{cloudHeight} seem to play an important role in the survival analysis. Cox and ANOVA 
tests (see listing \ref{lst_cox} for extracts of the \texttt{R} code) are used to select the significant 
covariates in order to preceed with analysis. The result of this ANOVA analysis using \texttt{isWarm, temperature, 
windSpeed} and \texttt{cloudHeight} covariates show a p-value smaller than 0.05, which indicates that we rejext 
the null hypotheses. The difference between the two models are significant and more covariates are necessary to 
explain the data. 

\begin{figure}[H]
\centering
\includegraphics[scale=0.6]{figures/correlationMatrix.png}
\caption{Correlation matrix of the covariates used in this study}
\label{fig_xcorr}
\end{figure}

\begin{lstlisting}[language=R, caption={Cox and ANOVA analysis on a selection of covariates}, captionpos=b, label={lst_cox}]
MtempWarmWindCloud <- coxph(Surv(start, stop, event)~temperature + isWarm + strata(season) + windSpeed + cloudHeight, data=data)

Mall <- coxph(Surv(start, stop, event)~isWarm + temperature + windSpeed + pressure + humidity + visibility + nebulosity + cloudHeight + moonPhase + strata(season), data=data)

anova(MtempWarmWindCloud, Mall)
\end{lstlisting}

Stepwise model selection is applied to the dataset using the AIC (Akaike Information Criterion) to select the 
most informative covariates and find the simplest model that would best fit the data. Results of the R code are 
presented in the listing \ref{lst_aic} and show that the highly significant covariates (with p-values $<0.01$) 
are : \texttt{longitude, isWarm, temperature, humidity, visibility, windSpeed}, and \texttt{cloudHeight}. 
\texttt{pressure} is a significant covariate (with p-value $<0.05$). \texttt{Nebulosity, latitude} and 
\texttt{moonPhase} covariates show p-values greater than 0.05 and are not significant covariates. The model 
considering only the highly significant model is named \texttt{Mbest}. ANOVA test is performed to compare 
\texttt{Mbest} model and the model considering all the covariates. Results show a p-value much greater than 0.05 
showing that we reject H1 hypothesis. Both models are significantly equals and the covariates of \texttt{Mbest} 
are sufficient to explain the data.\\ 
\newpage

\begin{lstlisting}[language=R, caption={Extracts of the output of the R code of the stepwise variable selection.}, captionpos=b, label={lst_aic}]
coxph(formula = Surv(start, stop, event) ~ latitude + longitude + 
    isWarm + temperature + windSpeed + pressure + humidity + 
    visibility + nebulosity + cloudHeight + moonPhase, data = data)

  n= 641, number of events= 376 

                coef exp(coef) se(coef)       z Pr(>|z|)    
latitude     0.04566   1.04671  0.08351   0.547 0.584588    
longitude    0.16842   1.18343  0.06528   2.580 0.009880 ** 
isWarm      -2.07290   0.12582  0.23488  -8.825  < 2e-16 ***
temperature -1.18656   0.30527  0.09937 -11.941  < 2e-16 ***
windSpeed   -0.19953   0.81912  0.05703  -3.499 0.000467 ***
pressure    -0.14994   0.86076  0.07606  -1.971 0.048687 *  
humidity    -0.40031   0.67011  0.09945  -4.025 5.69e-05 ***
visibility  -0.22837   0.79583  0.06243  -3.658 0.000254 ***
nebulosity  -0.10825   0.89740  0.06749  -1.604 0.108716    
cloudHeight  0.22859   1.25683  0.08634   2.648 0.008106 ** 
moonPhase    0.04129   1.04216  0.05335   0.774 0.438883   
\end{lstlisting}

The estimated Hazard Ratios of each covariates are presented in figure \ref{fig_HR} for the highly statistically 
significant covariates. \texttt{CloudHeight and longitude} covariate shows a harzard ratio greater than 1 
(HR=1.24 and 1.17 respectively) which indicates that the hazard increases when the cloud height and longitude 
increases. The thicker the cloud layer, the higher the risk.

Covariate \texttt{isWarm} shows the strongest effects with a HR value or 0.11. Event with warm days before the 
observation of a white stork have greatly reduced hazard.

Relatively strong effect is associated to the covariate \texttt{temperature} with a HR value of 0.32. Higher 
temperature is associated with considerable lower hazard.

With a hazard ratio close the 1 for the \texttt{pressure, widSpeed, humidity} and \texttt{visibility} covariates, 
these weather parameters seem to have moderate effect on the modelisation of the event (\textit{i.e.} the 
observation of the white stork).

\begin{figure}[H]
\centering
\includegraphics[scale=0.7]{figures/hazardRatios.png}
\caption{Hazard ratios plots of significant covariates}
\label{fig_HR}
\end{figure}


% ..................................................................................................
\subsection{Evaluation of time-dependent risks}
Schoenfeld residuals test is performed to verify the proportional hazards (PH) assumption that requires that the 
hazard ratios remain constant over time. A correlation of the scaled Schoenfeld residuals with time 
(\textit{i.e.} a variation of $\beta$ over time for each covariate) suggests that the effect of the covariates 
would change over time.

Results of the Schoenfeld test for the \texttt{Mbest} model are presented in listing \ref{lst_zph} and figure 
\ref{fig_zph} and show that the PH assumption is violated for all covariates but \texttt{visibility}. Indeed, 
p-values of \texttt{longitude, latitude, pressure, humidity, temperature, windSpeed} and \texttt{isWarm} 
covariates are highly statistically significant (p-value $<0.01$). \texttt{cloudHeight} covariate show a 
statistically significant violation of the PH assumption (p-value $<0.05$). 


\begin{lstlisting}[language=R, caption={Output of the R code for Schoenfeld test (\texttt{cox.zph(Mbest))}.}, captionpos=b, label={lst_zph}]
             chisq  p-value
longitude   18.55  1 1.7e-05 (< 0.01)
latitude    24.64  1 6.9e-07 (< 0.01)
pressure     9.55  1 0.00199 (< 0.01)
temperature 13.98  1 0.00019 (< 0.01)
isWarm       7.46  1 0.00629 (< 0.01)
windSpeed    7.05  1 0.00793 (< 0.01)
humidity    16.21  1 5.7e-05 (< 0.01)
visibility   1.57  1 0.21018 (> 0.05)
cloudHeight  5.49  1 0.01915 (< 0.05)
\end{lstlisting}

\begin{figure}[H]
\centering
\includegraphics[scale=0.7]{figures/cox_zph.png}
\caption{Scaled Schoenfeld residuals over time for each covariate of the \texttt{Mbest} model.}
\label{fig_zph}
\end{figure}

Figure \ref{fig_zph} shows a clear violation of the PH asusmption for \texttt{isWarm} covariate. Indeed, the 
sharp and strong variation of Beta with time shows a clear time-dependancy of the covariate. Moreover, the 
distribution of Beta(t) for \texttt{temperature} and \texttt{windSpeed} shows a slight upward slope while the  
distrubution for \texttt{humidity} and \texttt{latitude} show a slight downward slope. Both indicating 
time-dependancy of the covariates. 

% ..................................................................................................
\subsection{Residuals between observed and expected events}
Results of the Schoenfeld test show that the hazard ratios are not constant over time and suggest time variation 
of many covariates. Time-dependency are added to the cox function as presented in the listing \ref{lst_tt}. 
Several functions have been tested and the \texttt{x/t} function presents better results which are presented 
in this report. More details on this analysis are presented in the R code file. 

\begin{lstlisting}[language=R, caption={Modelisation of time variations.}, captionpos=b, label={lst_tt}]
Mbest_tt <- coxph(Surv(start, stop, event)~temperature + tt(temperature) + 
      isWarm + tt(isWarm) +  windSpeed + tt(windSpeed) + humidity + tt(humidity) + 
      visibility + cloudHeight + strata(season), data=data, tt=function(x,t,...) x/t)
\end{lstlisting}

Martingale residuals, showing the difference between the observed number of events and the expected number of 
events for \texttt{Mbest} (left) and \texttt{Mbest\_tt} (right) models are presented in figure \ref{fig_residuals}.
Figure \ref{fig_hist_residuals} presents their histograms and boxplot distributions. Martingale residuals are 
partially taken into account in the "Time-varying best model". Indeed, the dispersion of the residuals decreases 
and most of the residuals are closer to 0. The number of censored observations (over-predicted risk), with 
residuals lower than -1, are better constraints by the time-varying model. However, the model still fails to 
predict some measurements and did not fully captured the hazard variation. Indeed, many measurements residuals 
remain close to 1 which indicates that these events have underestimated risks by the model. 

\begin{figure}[H]
\centering
\includegraphics[scale=0.65]{figures/martingale.png}
\caption{Martingale analysis on \texttt{Mbest} (left) and \texttt{Mbest\_tt} (right) models. Red lines show 
a line smoothing of the scattered residuals data.}
\label{fig_residuals}
\end{figure}

\begin{figure}[H]
\centering
\includegraphics[scale=0.8]{figures/hist_residuals_boxplot.png}
\caption{Histograms of the Martingale residuals on \texttt{Mbest} (left) and \texttt{Mbest\_tt} (center) models. Left: boxplot representation of the residuals for both models.}
\label{fig_hist_residuals}
\end{figure}

