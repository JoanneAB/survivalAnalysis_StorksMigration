% ---------------------------------------------------------------------------------------------------------------
\section{Discussion}

% ...............................................................................................................
\subsection{Time-dependency of the covariates}
Despite the introduction of time-varying coefficients in the \texttt{Mbest\_tt} model, some temporal variation 
remains unexplained, particularly for \texttt{isWarm} covariate. The Schoenfeld residuals show sharp and 
non-linear variations in the effect of \texttt{isWarm} between $\sim80-120$ days that corresponds to the Spring 
season that is the key season for warm temperatures and migratory returns. \texttt{x/t} transformation is 
insufficient to fully capture temporal pattern. The binary nature of \texttt{isWarm} covariate may be 
a limitation and continuous measures of cumulative warm-days might better reflect the mechanisms behind white 
stork returns.

\texttt{isWarm} covariate captures informations on whether four consecutive warm days with a morning temperature 
greater than 8\textdegree C did occur before the event. Changing the number of consecutive days or the temperature 
from 8\textdegree C to 10\textdegree C did not meaningfull improved the model fit. This suggests that the 
threshold matters less than the overall weather conditions of the seasons. Alternative formulations to describe a 
warmer-than-usual season should be explored.\\

The stratification variable \texttt{season} did not significantly improve the modeling, suggesting that seasonal 
variation is already partially captured by the weather covariates.

% ...............................................................................................................
\subsection{Limitations}
The dataset used in this study show some limitations and could be extended and refined to better analyse the 
migratory return dates of the white storks in Alsace. 

The dataset only relies on weather records at two stations (Strasbourg and Mulhouse), which may not fully 
represent spatial variability across Alsace's sub-regions. Additional records at weather stations distributed 
across the region would help capture local meteorological variability more accurately.

Missing values represent $\sim$2.6\% of the dataset which may not be negligeable regarding the small amount of 
data collected. These missing values were replaced with covariate means, which may introduce bias to the dataset.

