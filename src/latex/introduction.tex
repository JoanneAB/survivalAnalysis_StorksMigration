% ---------------------------------------------------------------------------------------------------------------
\section{Introduction}
Considered as the symbol of Alsace (South-East region of France), the white stork (figure \ref{fig_cigogne}) 
returns to Alsace every spring from winter migration in Africa. Every spring, the return of the white stork to 
Alsace, marks an important events of the season. 

\begin{figure}[H]
\centering
\includegraphics[scale=1]{figures/cigogne.png}
\caption{Picture of the white stork (left, \cite{cigogne}), illustration of Hansi "La Cigogne de notre Village est revenue" (right).} 
\label{fig_cigogne}
\end{figure}

From a scientific perspective, the timing of migratory returns is known to be influenced by a combination of 
environmental factors, including temperature, wind conditions, and cloud coverage. Understanding and modelling 
these factors is of growing importance in the context of climate change as seasonal weather patterns may 
significantly alter bird migration phenology.

This study applies survival analysis to model the time until the first white stork observation of the year in 
Alsace, using data collected across the region. Survival times are defined as the number of days from the start of 
the year until the first stork is observed. Weather data as well as lunar phase information are used as covariates 
to explain the variability in return dates.

A Cox proportional hazards model is fitted to the data, and the proportional hazards assumption is evaluated and 
addressed through the introduction of time-varying coefficients. 
