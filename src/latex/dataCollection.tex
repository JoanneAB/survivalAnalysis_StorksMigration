% ---------------------------------------------------------------------------------------------------------------
\section{Data collection and data processing}

The aim of this study is to analyse the time until the first observation of the year of the first white stork in 
Alsace region, France after their return from migration in Africa. To construct the dataset, the geographical 
region of interest is divided into several sub-regions (figure \ref{fig_map}). Survival times are the number of 
days, for each year in each sub-region, until the first stork has been observed. Additional data such as weather 
parameters or moon phase are added to the dataset. 

\begin{figure}[H]
\centering
\includegraphics[scale=0.6]{figures/Figure_1.png}
\caption{Map of Alsace, France region. Black squares represent the sub-regions in which the data for each first observations of the white storks are collected. Red crosses represent the meteorological stations.}
\label{fig_map}
\end{figure}

% ---------------------------------------------------------------------------------------------------------------
\subsection{Data collection}
% ...............................................................................................................
\subsubsection{Observations of the white stork}
Dates and locations of the observations of white storks in France are collected from the Global Biodiversity 
Information Facility (GBIF) website using the latin name: Ciconia ciconia (Linnaeus, 1758) \citep{gbif}. The GBIF 
website compiles a total of 1990 distincts datasets. This collected dataset of white stork observations consists 
of 1883 088 observations (latitude, longitude and date) from January 1960 to October 2025.

% ...............................................................................................................
\subsubsection{Weather data}
Weather data in Alsace are collected from the "Observation m\'et\'eorologique historiques France" \citep{synop}. 
Data are recorded every three hours from 1996 at two stations : Strasbourg and Mulhouse. Recorded data are 
temperature, wind speed, pressure, humidity, horizonatal visibility, nebulosity and cloud height.

% ...............................................................................................................
\subsubsection{Lunar phases}
The moon is known to be a compass and landmark for the birds migrations. The fraction of illuminated moon could 
play an important role on the date of observation of the first stork in Alsace. Moon illumination fractions are 
collected from the Astronomical Applications Department \citep{moon}.

% ---------------------------------------------------------------------------------------------------------------
\subsection{Covariates engineering}
% ...............................................................................................................
\subsubsection{Time-dependent covariate}
Weather and more especially temperatures play an important role on the migratory return dates of the white 
storks. Indeed, a period of several warm days would lead to an increase in potential stork's food (insects, small 
mammals...) and an earlier return of the storks.

For each entry of the dataset, a new covariate is introduced to the dataset: 'isWarm'. This boolean covariate 
indicates whether four consecutive warm days (\textit{i.e.} with a morning temperature greater or equal to 
8\textdegree C) did occur the same year, before the first observation of the stork in the sub-region 
(\textit{i.e.} the event). Regarding the time-dependency of this covariate, if warm days have been recorded, the 
entry of the dataset is split into two entries. Table \ref{table_isWarm} presents a subset of the dataset where 
the entry with \texttt{id=1} do not show four consecutive warm days (\texttt{isWarm=0}) before the event at day 
88. The entry with \texttt{id=2} is divided into two lines with a first line that represents the time to the 
occurence of four consecutive warm days (at day 79, \texttt{isWarm=1}). The second line is the time between the 
warm days and the event (observation of the first stork). 


\begin{table}[H]
\centering \begin{tabular}{cccccc}
id & start & stop & event & isWarm & ...\\\hline\hline
1  & 0     & 88   & 1     & 0      & ...\\
2  & 0     & 79   & 0     & 1      & ...\\
2  & 79    & 89   & 1     & 1      & ...\\
\end{tabular}
\caption{}
\label{table_isWarm}
\end{table}

A new variable is added to the dataset to stratified by season (listing \ref{lst_season}), allowing the 
survival function to vary freely across seasons (listing \ref{lst_season}).

\begin{lstlisting}[language=R, caption={New stratified variable}, captionpos=b, label={lst_season}]
data$season <- factor(quarters(as.Date(data$stop)))
\end{lstlisting}

% ...............................................................................................................
\subsubsection{Covariate truncations}
Truncations on covariates are done to reduce the range of possible values, avoid extrem values and reduce the 
number of outliers.

White storks typically fly at altitudes between 500 and 1500m. Although some observations report altitudes of up 
to 4000m, these remain rare. The covariate \texttt{cloudHeight} was truncated to 3000 as the effect of higher 
values might be very limited.  

White storks experience difficulties in flying with winds stronger than 3-5 m/s. The covariate 
\texttt{windSpeed} was therefore truncated at 5m/s.

% ...............................................................................................................
\subsubsection{Covariates pre-processing}
Each non-boolean covariates (except \texttt{id}, \texttt{start}, \texttt{stop} and \texttt{event} covariates) 
are standardized in order to have a zero-mean and a standard deviation equals to 1.

Missing values, that corresponds to 276 values out of 10~605 total entries, are replaced by the mean of its 
covariate (\textit{i.e.} 0). Missing values are mostly related to missing weather measurements.

Figure \ref{fig_outliers} presents the distribution of \texttt{temperature/stop} ratio of covariates as a 
function of \texttt{stop} and show some outliers with values greater than 0.1 and smaller than -0.2. These 
entries are removed from the database. 

\begin{figure}[H]
\centering
\includegraphics[scale=0.6]{figures/remove_outliers.png}
\caption{Distribution of \texttt{temperature/stop} as a function of \texttt{stop}. Horizontal dashed lines outline the presence of outliers.}
\label{fig_outliers}
\end{figure}

% ---------------------------------------------------------------------------------------------------------------
\subsection{The final dataset}
A total of 641 observations were collected, each with 16 covariates : \texttt{start, stop, event, isWarm, 
latitude, longitude, temperature, windSpeed, pressure, humidity, visibility, nebulosity, cloudHeight, moonPhase} 
and \texttt{season}. The distribution of the standardized covariates are presented in figure \ref{fig_data}. 
More informations and figures are available in the R markdown script.  

\begin{figure}[H]
\centering
\includegraphics[scale=0.6]{figures/boxplots.png}
\caption{Distribution of the standardized covariates.}
\label{fig_data}
\end{figure}

