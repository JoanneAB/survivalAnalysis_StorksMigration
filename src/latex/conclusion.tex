% ---------------------------------------------------------------------------------------------------------------
\section{Conclusion}
This study applied survival analysis to model the return dates of white storks (\textit{Ciconia ciconia}) to 
Alsace, France, using observational data of white storks combined with historical weather and lunar phases 
records.

The analysis identified six statistically significant covariates: \texttt{isWarm, temperature, humidity, 
visibility, windSpeed} and \texttt{cloudHeight}. Among these covariates, the occurrence of consecutive warm days 
(\textit{i.e.} \texttt{isWarm} boolean covariate) and ambient temperature showed the strongest effects, with 
hazard ratios well below 1, confirming that warmer early-season conditions are strongly associated with earlier 
stork arrivals. In contrast, increased cloud height was associated with a delayed first observation.

The proportional hazards assumption was found to be violated for most covariates, highlighting time-dependency 
of the weather conditions to stork migration. Introducing time-varying coefficients transformation partially 
addressed this issue, reducing residual dispersion and improving the fit. However, the model did not fully 
capture the hazard variation, suggesting that more temporal formulations require further investigation.

Overall, this work demonstrates that weather conditions are key factors of white stork return in Alsace. Going 
further, the model could be used with up-to-date weather data to forecast the arrival date of the first white 
stork in Alsace in the future.
